% THIS DOCUMENT IS FOLLOWS THE VOLERE TEMPLATE BY Suzanne Robertson and James Robertson
% ONLY THE SECTION HEADINGS ARE PROVIDED
%
% Initial draft from https://github.com/Dieblich/volere
%
% Risks are removed because they are covered by the Hazard Analysis
\documentclass[12pt]{article}

\usepackage{booktabs}
\usepackage{tabularx}
\usepackage{hyperref}
\hypersetup{
    bookmarks=true,         % show bookmarks bar?
      colorlinks=true,      % false: boxed links; true: colored links
    linkcolor=red,          % color of internal links (change box color with linkbordercolor)
    citecolor=green,        % color of links to bibliography
    filecolor=magenta,      % color of file links
    urlcolor=cyan           % color of external links
}
\usepackage{enumitem}
\newcommand{\lips}{\textit{Insert your content here.}}

%% Comments

\usepackage{color}

\newif\ifcomments\commentstrue %displays comments
%\newif\ifcomments\commentsfalse %so that comments do not display

\ifcomments
\newcommand{\authornote}[3]{\textcolor{#1}{[#3 ---#2]}}
\newcommand{\todo}[1]{\textcolor{red}{[TODO: #1]}}
\else
\newcommand{\authornote}[3]{}
\newcommand{\todo}[1]{}
\fi

\newcommand{\wss}[1]{\authornote{magenta}{SS}{#1}} 
\newcommand{\plt}[1]{\authornote{cyan}{TPLT}{#1}} %For explanation of the template
\newcommand{\an}[1]{\authornote{cyan}{Author}{#1}}

%% Common Parts

\newcommand{\progname}{SFWRENG 4G06 - Capstone Design Process}
\newcommand{\authname}{\textbf{Team 17, DomainX} \\
\\ Awurama Nyarko
\\ Haniye Hamidizadeh
\\ Fei Xie
\\ Ghena Hatoum             
}
\usepackage{hyperref}
    \hypersetup{colorlinks=true, linkcolor=blue, citecolor=blue, filecolor=blue,
                urlcolor=blue, unicode=false}
    \urlstyle{same}
                                


\begin{document}

\title{Software Requirements Specification for \progname: subtitle describing software} 
\author{\authname}
\date{\today}
	
\maketitle

~\newpage

\pagenumbering{roman}

\tableofcontents

~\newpage

\section*{Revision History}

\begin{tabularx}{\textwidth}{p{3cm}p{2cm}X}
\toprule {\textbf{Date}} & {\textbf{Version}} & {\textbf{Notes}}\\
\midrule
Date 1 & 1.0 & Notes\\
Date 2 & 1.1 & Notes\\
\bottomrule
\end{tabularx}

~\\

~\newpage
\section{Purpose of the Project}
\subsection{User Business}
\lips
\subsection{Goals of the Project}
\lips
\section{Stakeholders}
\subsection{Client}
\lips
\subsection{Customer}
\lips
\subsection{Other Stakeholders}
\lips
\subsection{Hands-On Users of the Project}
\lips
\subsection{Personas}
\lips
\subsection{Priorities Assigned to Users}
\lips
\subsection{User Participation}
\lips
\subsection{Maintenance Users and Service Technicians}
\lips

\section{Mandated Constraints}
\subsection{Solution Constraints}
\lips
\subsection{Implementation Environment of the Current System}
\lips
\subsection{Partner or Collaborative Applications}
\lips
\subsection{Off-the-Shelf Software}
\lips
\subsection{Anticipated Workplace Environment}
\lips
\subsection{Schedule Constraints}
\lips
\subsection{Budget Constraints}
\lips
\subsection{Enterprise Constraints}
\lips

\section{Naming Conventions and Terminology}
\subsection{Glossary of All Terms, Including Acronyms, Used by Stakeholders
involved in the Project}
\lips

\section{Relevant Facts And Assumptions}
\subsection{Relevant Facts}
\lips
\subsection{Business Rules}
\lips
\subsection{Assumptions}
\lips

\section{The Scope of the Work}
\subsection{The Current Situation}
\lips
\subsection{The Context of the Work}
\lips
\subsection{Work Partitioning}
\lips
\subsection{Specifying a Business Use Case (BUC)}
\lips

\section{Business Data Model and Data Dictionary}
\subsection{Business Data Model}
\lips
\subsection{Data Dictionary}
\lips

\section{The Scope of the Product}
\subsection{Product Boundary}
\lips
\subsection{Product Use Case Table}
\lips
\subsection{Individual Product Use Cases (PUC's)}
\lips

\section{Functional Requirements}
\subsection{Functional Requirements}
\lips

\section{Look and Feel Requirements}
\subsection{Appearance Requirements}
\lips
\subsection{Style Requirements}
\lips

\section{Usability and Humanity Requirements}
\subsection{Ease of Use Requirements}
\lips
\subsection{Personalization and Internationalization Requirements}
\lips
\subsection{Learning Requirements}
\lips
\subsection{Understandability and Politeness Requirements}
\lips
\subsection{Accessibility Requirements}
\lips

\section{Performance Requirements}
\subsection{Speed and Latency Requirements}
\lips
\subsection{Safety-Critical Requirements}
\lips
\subsection{Precision or Accuracy Requirements}
\lips
\subsection{Robustness or Fault-Tolerance Requirements}
\lips
\subsection{Capacity Requirements}
\lips
\subsection{Scalability or Extensibility Requirements}
\lips
\subsection{Longevity Requirements}
\lips

\section{Operational and Environmental Requirements}
\subsection{Expected Physical Environment}

\begin{enumerate}[label=\thesubsection-\arabic*]
\item The tool is a web-based application that will be used mainly by our team, supervisors, and potentially other researchers or domain experts.
\item It is expected to run on a standard desktop or laptop computer with a reliable internet connection, in a normal indoor setting such as an office, lab, or home workspace.
\item No specialized hardware or rugged equipment is required. A keyboard, mouse, or touchpad, and a modern web browser (e.g., Chrome, Firefox, Edge) are sufficient.
\end{enumerate}

\subsection{Wider Environment Requirements}
\begin{enumerate}[label=\thesubsection-\arabic*]
\item The application depends on a stable internet connection for retrieving data from public repositories (e.g., GitHub) and for loading the hosted web interface if deployed to the cloud.
\item It does not rely on any dedicated on-premises hardware.
\item The tool should work on the latest two to three major versions of common browsers such as Chrome, Firefox, and Edge. No special environmental conditions (lighting, noise, temperature) are expected to affect usability.
\end{enumerate}
\subsection{Requirements for Interfacing with Adjacent Systems}
The tool must be able to communicate with a few external systems and internal components to collect, store, and visualize data.

\begin{enumerate}[label=\thesubsection-\arabic*]

  \item Public Repository APIs (e.g., GitHub API): The tool must communicate with external repository services to automatically collect information about the selected neural-network libraries. For example, it needs to retrieve details such as how often the code is updated, the number of open or closed issues and pull requests, and the main programming languages used. This data will help evaluate qualities like maintainability, transparency, and overall project activity. The information must be received in a standard machine-readable format (e.g., JSON over HTTPS) whenever a user triggers a scan or during scheduled updates.

  \item Database (MySQL): The backend must store scores, rankings, and evidence collected from repositories.

  \item Visualization Component: The frontend must render charts and comparisons (e.g., using Chart.js) from the data served by the backend.

  \item All connections must use standard web technologies (HTTP/HTTPS, JSON) and require only basic authentication methods such as API tokens for secure access to external repository APIs.

\end{enumerate}

\subsection{Productization Requirements}
The tool will be delivered as an open-source web application hosted in the team’s public GitHub repository.
\begin{enumerate}[label=\thesubsection-\arabic*]
\item A clear README file must explain how to set up the backend (Python + Flask) and the frontend (React) using common package managers such as pip and npm.
\item For developers running the tool locally, the repository must include a requirements.txt file for Python packages and a database schema file so they can create the required tables.
\item When deployed on a cloud platform such as AWS or Google Cloud, users must be able to access the tool directly through a web URL without installing anything.
\item The application must also provide options to export results in formats such as CSV for data tables and PNG/PDF for visualizations.
\end{enumerate}
\subsection{Release Requirements}
\begin{enumerate}[label=\thesubsection-\arabic*]
\item The project should follow the official capstone timeline: an internal test release before the Proof-of-Concept (PoC) demonstration in \textbf{November}, a Revision~0 Demonstration in \textbf{Weeks~18--19}, and the Final Release (Revision~1) at \textbf{Week~26} along with the research paper and final dataset.
\item The Final Release (Revision 1) must incorporate feedback collected during the Revision 0 Demonstration, including usability improvements, bug fixes, and supervisor/TA-requested changes.
\item All releases must be published as GitHub Releases and include a short changelog describing the changes in each version.

  \item Releases must use a clear, descriptive version tag such as the \texttt{MAJOR.MINOR.PATCH} format (or an equally descriptive format):
    \begin{itemize}
        \item \textbf{MAJOR:} Increased only when a change breaks backward compatibility (e.g., a database schema change that makes older data unusable).
        \item \textbf{MINOR:} Increased when adding new features that remain fully compatible with previous versions.
        \item \textbf{PATCH:} Increased for bug fixes or small improvements that do not affect existing features.
    \end{itemize}

  Example version tags:
    \begin{itemize}
        \item \texttt{v0.1.0} → first working prototype for the Revision~0 Demonstration
        \item \texttt{v0.2.0} → adds a new feature such as exporting the table to a CSV file
        \item \texttt{v0.2.1} → fixes a small bug in the export feature (patch)
        \item \texttt{v1.0.0} → stable Final Release for Revision~1
    \end{itemize}
\end{enumerate}
\section{Maintainability and Support Requirements}
\subsection{Maintenance Requirements}
The tool is an open-source web application that will need occasional updates to fix bugs and to adapt if external APIs change.
\begin{enumerate}[label=\thesubsection-\arabic*]
\item All code must follow the project’s coding standards (PEP 8 for the Python backend; React + TypeScript style guide for the frontend).
\item Automated tools (such as Black, Flake8, and Pylint) must remain part of the workflow so new contributors can easily read and update the code.
\item Unit tests must be written for all new features and bug fixes. Developers should also run basic integration tests to make sure the full pipeline (API → database → visualization) still works after changes.
\item Test coverage must be tracked and reported to ensure that critical parts of the backend are being tested.
\item All Python and frontend packages must be listed in requirements.txt and package.json, with pinned versions so the same build can be reproduced.
\item The dependency list must be reviewed and updated at least once each semester to keep it current and secure.
\item Any changes to the dataset made through the interactive data table must be logged with a timestamp and the username so there is always a clear audit trail.
\item Most routine fixes, such as small UI tweaks or bug fixes, should be finished within a couple of days. Larger updates, such as adding a new metric or a new visualization, are expected to take about one to two weeks.
\end{enumerate}
\subsection{Supportability Requirements}
The tool is intended to be mostly self-supporting since it will be used primarily by our team, the supervisors, and potentially other researchers in the future.
\begin{enumerate}[label=\thesubsection-\arabic*]
\item The README file must include step-by-step instructions so that a new developer can set up the backend (Flask + MySQL) and frontend (React) locally, or deploy it to a cloud platform (such as AWS or Google Cloud), in a consistent and repeatable way. A new developer should be able to follow these instructions and have the application running within about two hours.
\item The repository must always include an up-to-date guide for installation, setup, the data-collection workflow, using the interactive data table, visualization, and exporting results.
\item The backend must include logging to record API failures (such as rate-limit errors or unexpected data formats), database issues, and runtime exceptions so that maintainers can troubleshoot problems quickly.
\item Users should use the GitHub Issues page to report bugs or ask questions. No printed manual will be needed; all documentation will remain online in the repository.
\end{enumerate}
\subsection{Adaptability Requirements}
\begin{enumerate}[label=\thesubsection-\arabic*]
\item The tool must remain flexible so it can grow with the project and adapt to future needs.
\item The MySQL database must be set up so that if a new quality criterion needs to be tracked, it can be added by creating a new column, updating the data-collection script, and adjusting the UI without having to redesign the whole system.
\item The data-collection module must allow new data sources (for example, another code-hosting site or a different metrics service) to be added with minimal extra code and without disrupting the existing GitHub integration.
\item The tool must be able to run on standard operating systems(Windows, macOS, and Linux) by using widely supported technologies (such as Python, Node.js, and MySQL), and by avoiding any system-specific code.
\item The system must be designed in a modular way so that parts like data collection, storage, and visualization remain separate. This makes it easier to add new features or update one part without affecting the rest of the tool.
\end{enumerate}
\section{Security Requirements}
\subsection{Access Requirements}
\lips
\subsection{Integrity Requirements}
\lips
\subsection{Privacy Requirements}
\lips
\subsection{Audit Requirements}
\lips
\subsection{Immunity Requirements}
\lips

\section{Cultural Requirements}
\subsection{Cultural Requirements}
\lips

\section{Compliance Requirements}
\subsection{Legal Requirements}
\lips
\subsection{Standards Compliance Requirements}
\lips

\section{Open Issues}
\lips

\section{Off-the-Shelf Solutions}
\subsection{Ready-Made Products}
\lips
\subsection{Reusable Components}
\lips
\subsection{Products That Can Be Copied}
\lips

\section{New Problems}
\subsection{Effects on the Current Environment}
\lips
\subsection{Effects on the Installed Systems}
\lips
\subsection{Potential User Problems}
\lips
\subsection{Limitations in the Anticipated Implementation Environment That May
Inhibit the New Product}
\lips
\subsection{Follow-Up Problems}
\lips

\section{Tasks}
\subsection{Project Planning}
\lips
\subsection{Planning of the Development Phases}
\lips

\section{Migration to the New Product}
\subsection{Requirements for Migration to the New Product}
\lips
\subsection{Data That Has to be Modified or Translated for the New System}
\lips

\section{Costs}
\lips
\section{User Documentation and Training}
\subsection{User Documentation Requirements}
\lips
\subsection{Training Requirements}
\lips

\section{Waiting Room}
\lips

\section{Ideas for Solution}
\lips

\newpage{}
\section*{Appendix --- Reflection}

The purpose of reflection questions is to give you a chance to assess your own
learning and that of your group as a whole, and to find ways to improve in the
future. Reflection is an important part of the learning process.  Reflection is
also an essential component of a successful software development process.  

Reflections are most interesting and useful when they're honest, even if the
stories they tell are imperfect. You will be marked based on your depth of
thought and analysis, and not based on the content of the reflections
themselves. Thus, for full marks we encourage you to answer openly and honestly
and to avoid simply writing ``what you think the evaluator wants to hear.''

Please answer the following questions.  Some questions can be answered on the
team level, but where appropriate, each team member should write their own
response:


\begin{enumerate}
  \item What went well while writing this deliverable? 
  \item What pain points did you experience during this deliverable, and how did
  you resolve them?
  \item How many of your requirements were inspired by speaking to your
  client(s) or their proxies (e.g. your peers, stakeholders, potential users)?
  \item Which of the courses you have taken, or are currently taking, will help
  your team to be successful with your capstone project.
  \item What knowledge and skills will the team collectively need to acquire to
  successfully complete this capstone project?  Examples of possible knowledge
  to acquire include domain specific knowledge from the domain of your
  application, or software engineering knowledge, mechatronics knowledge or
  computer science knowledge.  Skills may be related to technology, or writing,
  or presentation, or team management, etc.  You should look to identify at
  least one item for each team member.
  \item For each of the knowledge areas and skills identified in the previous
  question, what are at least two approaches to acquiring the knowledge or
  mastering the skill?  Of the identified approaches, which will each team
  member pursue, and why did they make this choice?
\end{enumerate}


\end{document}