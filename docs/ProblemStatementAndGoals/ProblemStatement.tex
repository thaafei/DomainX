\documentclass{article}

\usepackage{tabularx}
\usepackage{booktabs}

\title{Problem Statement and Goals\\\progname}

\author{\authname}

\date{September 20, 2025}

%% Comments

\usepackage{color}

\newif\ifcomments\commentstrue %displays comments
%\newif\ifcomments\commentsfalse %so that comments do not display

\ifcomments
\newcommand{\authornote}[3]{\textcolor{#1}{[#3 ---#2]}}
\newcommand{\todo}[1]{\textcolor{red}{[TODO: #1]}}
\else
\newcommand{\authornote}[3]{}
\newcommand{\todo}[1]{}
\fi

\newcommand{\wss}[1]{\authornote{magenta}{SS}{#1}} 
\newcommand{\plt}[1]{\authornote{cyan}{TPLT}{#1}} %For explanation of the template
\newcommand{\an}[1]{\authornote{cyan}{Author}{#1}}

%% Common Parts

\newcommand{\progname}{SFWRENG 4G06 - Capstone Design Process}
\newcommand{\authname}{\textbf{Team 17, DomainX} \\
\\ Awurama Nyarko
\\ Haniye Hamidizadeh
\\ Fei Xie
\\ Ghena Hatoum             
}
\usepackage{hyperref}
    \hypersetup{colorlinks=true, linkcolor=blue, citecolor=blue, filecolor=blue,
                urlcolor=blue, unicode=false}
    \urlstyle{same}
                                


\begin{document}

\maketitle

\begin{table}[hp]
\caption{Revision History} \label{TblRevisionHistory}
\begin{tabularx}{\textwidth}{llX}
\toprule
\textbf{Date} & \textbf{Developer(s)} & \textbf{Change}\\
\midrule
2025-09-20 & Awurama Nyarko, Fiona Xie, Ghena Hatoum, Fatemeh Hamidizadeh & Initial draft of Problem Statement (Problem, Inputs/Outputs, Stakeholders, Environment, Goals, Stretch, Extras)\\
\bottomrule
\end{tabularx}
\end{table}

\section{Problem Statement}

% \wss{You should check your problem statement with the
% \href{https://github.com/smiths/capTemplate/blob/main/docs/Checklists/ProbState-Checklist.pdf}
% {problem statement checklist}.} 

% \wss{You can change the section headings, as long as you include the required
% information.}

\subsection{Problem}
Open-source neural network (NN) libraries---including training frameworks
(e.g., PyTorch, TensorFlow), inference frameworks (e.g., ONNX Runtime),
and companion tooling (e.g., data loaders, visualization utilities)---are central
to academic research and production AI systems. Despite their importance,
there is currently no transparent, reproducible, and domain-tailored way to
assess the state of software-development practice across these libraries.

Existing assessments are typically ad hoc: spreadsheet-based scoring and
manual pairwise comparisons (e.g., AHP) that are difficult to trace, with
evidence scattered across READMEs, documentation, tests, and CI
pipelines. This creates challenges for: (1) users (researchers and engineers)
who cannot easily compare libraries on qualities such as installability,
maintainability, understandability, usability, transparency, robustness,
reliability, and reproducibility; (2) maintainers, who lack cross-project
feedback on practice gaps; and (3) researchers, who lack a consistent
methodology and artifact trail suitable for replication or meta-analysis.

This project addresses these issues by adapting and applying a rigorous,
repeatable state-of-practice methodology to the NN-library domain, and
replacing fragile spreadsheets with a more traceable, auditable toolset that
streamlines evidence collection, scoring, and ranking. The outcome will be a
defensible comparison across software-engineering qualities and actionable
recommendations for practitioners and maintainers.


\subsection{Inputs and Outputs}
\paragraph{Inputs}
\begin{itemize}
  \item Curated set of open-source NN libraries that meet inclusion criteria
  (public repositories with source, installation docs, runnable examples/tutorials).
  \item Public artifacts: source code, documentation, tutorials, tests, CI configs,
  issue trackers, release notes.
  \item Repository measures: commit/issue activity, language breakdown,
  lines-of-code counts, collected via automated scripts.
  \item Measurement template responses: structured question bank per quality,
  completed via automation and human inspection.
  \item Decision-model configuration: weights over qualities (installability,
  maintainability, etc.) using a method such as AHP for reproducible rankings.
  \item Optional practitioner/maintainer interviews, if time permits.
\end{itemize}

\paragraph{Outputs}
\begin{itemize}
  \item Traceable dataset: structured table recording evidence and normalized
  scores per software quality.
  \item Rankings and sensitivity analyses: transparent results with full evidence trail,
  plus stability testing under weight changes.
  \item Domain insights: summarized patterns, practice gaps, correlations across projects.
  \item Recommendations: prioritized, evidence-backed suggestions for maintainers and users.
  \item Lightweight toolset: simple replacement for spreadsheets to collect evidence,
  apply scoring, and export results (CSV/PDF).
  \item Research deliverables: (i) an academic-style report documenting methods,
  dataset, findings; (ii) an open data/software package (versioned with changelog)
  for reproducibility and reuse.
\end{itemize}

% \wss{Characterize the problem in terms of ``high level'' inputs and outputs.  
% Use abstraction so that you can avoid details.}

\subsection{Stakeholders}
\begin{itemize}
  \item Supervisors and domain experts (primary): define scope, validate inclusion criteria,
  review methodology, and vet results.
  \item NN library maintainers and contributors (primary): benefit from diagnostics
  and recommendations; possible interview participants.
  \item Practitioners (researchers, engineers, data scientists) (primary): need trustworthy
  comparisons to select or advocate for libraries.
  \item Software engineering and RSE research community (secondary): gains reusable
  methodology, dataset, and toolset.
  \item Course instruction team (secondary): evaluates rigor, scope fit, and completeness
  of methodology and reporting.
\end{itemize}


\subsection{Environment}

\wss{Hardware and Software Environment}

\paragraph{Assumptions}
\begin{itemize}
  \item Target libraries expose public repositories with source code.
  \item Installation documentation is available.
  \item Software qualities can be evaluated from surface artifacts within the project timeline.
  \item Stakeholders are available for periodic reviews.
\end{itemize}

\paragraph{Constraints}
\begin{itemize}
  \item Fixed academic timeline.
  \item Limited GPU access.
  \item Private or internal artifacts excluded.
  \item Interviews optional and subject to limited availability.
  \item Rapid upstream changes mitigated via timestamped data collection.
\end{itemize}

\paragraph{Out of Scope}
\begin{itemize}
  \item Benchmarking model accuracy.
  \item Benchmarking runtime or throughput performance.
  \item Domain-specific fitness evaluations beyond software-engineering qualities.
\end{itemize}


\section{Goals}

\section{Stretch Goals}

\section{Extras}

% \wss{For CAS 741: State whether the project is a research project. This
% designation, with the approval (or request) of the instructor, can be modified
% over the course of the term.}

% \wss{For SE Capstone: List your extras.  Potential extras include usability
% testing, code walkthroughs, user documentation, formal proof, GenderMag
% personas, Design Thinking, etc.  (The full list is on the course outline and in
% Lecture 02.) Normally the number of extras will be two.  Approval of the extras
% will be part of the discussion with the instructor for approving the project.
% The extras, with the approval (or request) of the instructor, can be modified
% over the course of the term.}

\newpage{}

\section*{Appendix --- Reflection}

% \wss{Not required for CAS 741}

The purpose of reflection questions is to give you a chance to assess your own
learning and that of your group as a whole, and to find ways to improve in the
future. Reflection is an important part of the learning process.  Reflection is
also an essential component of a successful software development process.  

Reflections are most interesting and useful when they're honest, even if the
stories they tell are imperfect. You will be marked based on your depth of
thought and analysis, and not based on the content of the reflections
themselves. Thus, for full marks we encourage you to answer openly and honestly
and to avoid simply writing ``what you think the evaluator wants to hear.''

Please answer the following questions.  Some questions can be answered on the
team level, but where appropriate, each team member should write their own
response:


\begin{enumerate}
    \item What went well while writing this deliverable? 
    
    \textbf{Awurama:} I found that structuring the problem into the four subsections 
    (Problem, Inputs/Outputs, Stakeholders, Environment) went smoothly. Having my 
    draft already written in a google doc made it much easier to adapt to LaTeX. 
    
    \item What pain points did you experience during this deliverable, and how
    did you resolve them?

    \textbf{Awurama:} My main challenge was getting familiar with the LaTeX template 
    and re-adjusting to the GitHub workflow after not using it for a while. I overcame this 
    by researching LaTeX best practices and following the problem statement writing checklist 
    to make sure I stayed on track.  

    \item How did you and your team adjust the scope of your goals to ensure
    they are suitable for a Capstone project (not overly ambitious but also of
    appropriate complexity for a senior design project)?

    \textbf{Awurama:} As a team, we agreed to keep the project focused on assessing 
    software-engineering practices of neural network libraries, and to exclude accuracy 
    and performance benchmarking. This adjustment kept the scope realistic for the 
    Capstone timeline and ensured it matched the course expectations.  

\end{enumerate}  

\end{document}