\documentclass{article}

\usepackage{float}
\restylefloat{table}

\usepackage{booktabs}

\title{Team Contributions: Rev 0\\\progname}

\author{\authname}

\date{}

\input{../Comments}
\input{../Common}

\begin{document}

\maketitle

This document summarizes the contributions of each team member for the Rev 0
Demo.  The time period of interest is the time between the PoC demo and the Rev
0 demo; the contributions prior to the PoC are NOT included.

\section{Demo Plans}

% \wss{What will you be demonstrating}
For our Rev 0 demo, we will begin with a brief introduction
to the project and a short update on the research progress.
We will then walk through the main user flow end-to-end to
show that the core system components are integrated and
functioning. Starting from the Home page, the user navigates
to the Sign-In page and logs in. After signing in, the user
is taken to the Visualization page, where they can enter a
GitHub repository link and submit it for analysis. Our backend
retrieves repository data via the GitHub API, processes the
retrieved information, and returns results to the frontend.
We will demonstrate that outputs are successfully rendered
and that the system is usable end-to-end. The demo will also
highlight improvements since the PoC demo (e.g., more robust
data handling, clearer outputs, and a more consistent user
experience). If any planned functionality is not yet
implemented, we will state it explicitly and describe next
steps.

\section{Team Meeting Attendance}

% \wss{For each team member how many team meetings have they attended over the
% time period of interest.  This number should be determined from the meeting
% issues in the team's repo.  The first entry in the table should be the total
% number of team meetings held by the team.}

\begin{table}[H]
\centering
\begin{tabular}{ll}
\toprule
\textbf{Student} & \textbf{Meetings}\\
\midrule
Total & 4\\
Awurama & 4\\
Fei & 3\\
Ghena & 4\\
Haniye & 4\\
\bottomrule
\end{tabular}
\end{table}

Most communications have been made over Discord channels outside of meeting times.

\section{Supervisor/Stakeholder Meeting Attendance}

% \wss{For each team member how many supervisor/stakeholder team meetings have
% they attended over the time period of interest.  This number should be determined
% from the supervisor meeting issues in the team's repo.  The first entry in the
% table should be the total number of supervisor and team meetings held by the
% team.  If there is no supervisor, there will usually be meetings with
% stakeholders (potential users) that can serve a similar purpose.}

\noindent \textbf{Supervisor's Name: Dr. Spencer Smith}

\begin{table}[H]
\centering
\begin{tabular}{ll}
\toprule
\textbf{Student} & \textbf{Meetings}\\
\midrule
Total & 1\\
Awurama & 1\\
Fei & 1\\
Ghena & 1\\
Haniye & 1\\
\bottomrule
\end{tabular}
\end{table}

% \wss{If needed, an explanation for the counts can be provided here.}

\section{Lecture Attendance}

% \wss{For each team member how many lectures have they attended over the time
% period of interest.  This number should be determined from the lecture issues in
% the team's repo. You can find the number of lectures in the time period of
% interest by looking at the
% \href{https://calendar.google.com/calendar/u/0/embed?src=rnboqiaki1k2la7rpu3bn0um58@group.calendar.google.com&ctz=America/Toronto}
% {Google calendar} for the capstone course.}

% \wss{NOTE: There will be approximately 1 lecture between the POC and Rev0 demos}

\begin{table}[H]
\centering
\begin{tabular}{ll}
\toprule
\textbf{Student} & \textbf{Lectures}\\
\midrule
Total & 1\\
Awurama & 0\\
Fei & 1\\
Ghena & 1\\
Haniye & 0\\
\bottomrule
\end{tabular}
\end{table}

% \wss{If needed, an explanation for the lecture attendance can be provided here.}

\section{TA Document Discussion Attendance}

% \wss{For each team member how many of the informal document discussion meetings
% with the TA were attended over the time period of interest.}

\noindent \textbf{TA's Name: Tanya Djavaherpour} 

\begin{table}[H]
\centering
\begin{tabular}{ll}
\toprule
\textbf{Student} & \textbf{Lectures}\\
\midrule
Total & 1\\
Awurama & 1\\
Fei & 1\\
Ghena & 1\\
Haniye & 1\\
\bottomrule
\end{tabular}
\end{table}

% \wss{If needed, an explanation for the attendance can be provided here.}

\section{Commits}

% \wss{For each team member how many commits to the main branch have been made
% over the time period of interest.  The total is the total number of commits for
% the entire team since the beginning of the term.  The percentage is the
% percentage of the total commits made by each team member.}

\begin{table}[H]
\centering
\begin{tabular}{lll}
\toprule
\textbf{Student} & \textbf{Commits} & \textbf{Percent}\\
\midrule
Total & 44 & 100\% \\
Awurama & 18 & 41\% \\
Fei & 6 & 14\% \\
Ghena & 0 & 0\% \\
Haniye & 20 & 45\% \\
\bottomrule
\end{tabular}
\end{table}

% \wss{If needed, an explanation for the counts can be provided here.  For
% instance, if a team member has more commits to unmerged branches, these numbers
% can be provided here.  If multiple people contribute to a commit, git allows for
% multi-author commits.}
\noindent\textit{Note:} These commit counts are not an accurate depiction
of overall contribution. Most development work for the tool
was completed on a separate branch and has not yet been merged
into \texttt{main}, so the \texttt{main} branch commit totals
underrepresent the actual work completed during this period.

\section{Issue Tracker}

% \wss{For each team member how many issues have they authored (including open and
% closed issues (O+C)) and how many have they been assigned (only counting closed
% issues (C only)) over the time period of interest.}

\begin{table}[H]
\centering
\begin{tabular}{lll}
\toprule
\textbf{Student} & \textbf{Authored (O+C)} & \textbf{Assigned (C only)}\\
\midrule
Awurama & 8 & 15 \\
Fei & 12 & 25 \\
Ghena & 43 & 4 \\
Haniye & 20 & 9 \\
\bottomrule
\end{tabular}
\end{table}

% \wss{If needed, an explanation for the counts can be provided here.}

\section{CICD}

% \wss{Say how CICD is used in your project}
We use GitHub Actions to support both documentation
and software development workflows. For documentation,
GitHub Actions automatically builds/compiles modified
\LaTeX{} files on pull requests and merges to \texttt{main}.
Members merge changes into a feature branch after comments
are addressed. When at least two team members have approved,
the feature branch changes are merged into \texttt{main}.\\

For the software workflow, active development takes place
in our \texttt{release} branch. When changes are pushed or
a pull request is opened to update \texttt{release} or merge
into \texttt{main}, GitHub Actions runs automated checks on pull requests
(e.g., the \texttt{Run Tests} workflow) before merging. These checks help ensure changes
integrate cleanly before merging. Once the \texttt{release}
branch is stable and reviewed, it is merged into \texttt{main}.
\section{Team Charter Trigger Items}

% \wss{Provide a summary of the quantified triggers identified in the team's
% charter.}
\subsection{Summary of Triggers}
\begin{enumerate}
\item{Attending every weekly Monday virtual meeting during lecture time when there are no lectures. Unless the whole team decides to cancel}
\item{Team members expected to attend all meetings on time, fully prepared and having reviewed relevant materials and the agenda for the meeting}
\item{All deliverables must meet the team's agreed-upon standards}
\item{Deliverables must be reviewed by all assigned team members through the "reviewed" confirmation on GitHub, before deliverables are ready to be merged.}
\item{Feedback on deliverables and revisions must be made within 24 hours}
\end{enumerate}

% \wss{Provide a list of any violations of the triggers.  If the team wishes, the
% violations can be summarized on aggregate, instead of naming specific team
% members.}
No present trigger violations.
% \wss{Provide a plan to address the violations.  This could include revising the
% triggers, if they are found to be too weak, strong or ambiguous.}

\section{Additional Productivity Metrics}
No additional productivity metric was used by the team.
% \wss{If your team has additional metrics of productivity, please feel free to
% add them to this report.}

\end{document}