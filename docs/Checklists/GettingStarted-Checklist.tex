\documentclass[12pt]{article}

\usepackage{enumitem}

\usepackage{amssymb}
\usepackage{amsfonts}
\usepackage{amsmath}

\usepackage{hyperref}
\hypersetup{colorlinks=true,
    linkcolor=blue,
    citecolor=blue,
    filecolor=blue,
    urlcolor=blue,
    unicode=false}
\urlstyle{same}

\newlist{todolist}{itemize}{2}
\setlist[todolist]{label=$\square$}
\usepackage{pifont}
\newcommand{\cmark}{\ding{51}}%
\newcommand{\xmark}{\ding{55}}%
\newcommand{\done}{\rlap{$\square$}{\raisebox{2pt}{\large\hspace{1pt}\cmark}}%
\hspace{-2.5pt}}
\newcommand{\wontfix}{\rlap{$\square$}{\large\hspace{1pt}\xmark}}

\begin{document}

\title{Getting Started Checklist}
\author{Spencer Smith}
\date{\today}

\maketitle

% Show an item is done by   \item[\done] Frame the problem
% Show an item will not be fixed by   \item[\wontfix] profit

The following checklist is relevant when starting a new project in the SE capstone course.

\begin{itemize}
  
\item Form team and select project
  \begin{todolist}
  \item Find like-minded team-mates with similar goals
  \item Select a project (see the list of
  \href{https://gitlab.cas.mcmaster.ca/courses/capstone/-/blob/main/ListOfPotentialProjects/PotentialProjects.pdf?ref_type=heads}
  {Potential Projects})
  \item Use the \href{https://github.com/smiths/capTemplate/} {capTemplate}
  template (green button).  \textbf{Do not fork the template repo and do not
  start your own repo from scratch.}
  \item Your project name (as used in your documentation, and in your GitHub
  repo link) identifies the project in some way.  The project name should not be
  a generic name, like \texttt{capstone}.
  \item Select initial extras
  \item Create a pull request to modify \texttt{TeamComposition.csv}
  \end{todolist}

\item Set-up initial repo
  \begin{todolist}
  \item Select ``use this template'', do not fork
  \item Select initial LICENSE
  \item Update README to at least say what the project is
  \end{todolist}

\end{itemize}

\end{document}
