\documentclass{article}

\usepackage{booktabs}
\usepackage{tabularx}

\title{Development Plan\\\progname}

\author{\authname}

\date{}

%% Comments

\usepackage{color}

\newif\ifcomments\commentstrue %displays comments
%\newif\ifcomments\commentsfalse %so that comments do not display

\ifcomments
\newcommand{\authornote}[3]{\textcolor{#1}{[#3 ---#2]}}
\newcommand{\todo}[1]{\textcolor{red}{[TODO: #1]}}
\else
\newcommand{\authornote}[3]{}
\newcommand{\todo}[1]{}
\fi

\newcommand{\wss}[1]{\authornote{magenta}{SS}{#1}} 
\newcommand{\plt}[1]{\authornote{cyan}{TPLT}{#1}} %For explanation of the template
\newcommand{\an}[1]{\authornote{cyan}{Author}{#1}}

%% Common Parts

\newcommand{\progname}{SFWRENG 4G06 - Capstone Design Process}
\newcommand{\authname}{\textbf{Team 17, DomainX} \\
\\ Awurama Nyarko
\\ Haniye Hamidizadeh
\\ Fei Xie
\\ Ghena Hatoum             
}
\usepackage{hyperref}
    \hypersetup{colorlinks=true, linkcolor=blue, citecolor=blue, filecolor=blue,
                urlcolor=blue, unicode=false}
    \urlstyle{same}
                                


\begin{document}

\maketitle

\begin{table}[hp]
\caption{Revision History} \label{TblRevisionHistory}
\begin{tabularx}{\textwidth}{llX}
\toprule
\textbf{Date} & \textbf{Developer(s)} & \textbf{Change}\\
\midrule
Date1 & Name(s) & Description of changes\\
Date2 & Name(s) & Description of changes\\
... & ... & ...\\
\bottomrule
\end{tabularx}
\end{table}

\newpage{}

\wss{Put your introductory blurb here.  Often the blurb is a brief roadmap of
what is contained in the report.}

\wss{Additional information on the development plan can be found in the
\href{https://gitlab.cas.mcmaster.ca/courses/capstone/-/blob/main/Lectures/L02b_POCAndDevPlan/POCAndDevPlan.pdf?ref_type=heads}
{lecture slides}.}

\section{Confidential Information?}

\wss{State whether your project has confidential information from industry, or
not.  If there is confidential information, point to the agreement you have in
place.}

\wss{For most teams this section will just state that there is no confidential
information to protect.}
\section{IP to Protect}

\wss{State whether there is IP to protect.  If there is, point to the agreement.
All students who are working on a project that requires an IP agreement are also
required to sign the ``Intellectual Property Guide Acknowledgement.''}

\section{Copyright License}

\wss{What copyright license is your team adopting.  Point to the license in your
repo.}

\section{Team Meeting Plan}

\wss{How often will you meet? where?}

\wss{If the meeting is a physical location (not virtual), out of an abundance of
caution for safety reasons you shouldn't put the location online}

\wss{How often will you meet with your industry advisor?  when?  where?}

\wss{Will meetings be virtual?  At least some meetings should likely be
in-person.}

\wss{How will the meetings be structured?  There should be a chair for all meetings.  There should be an agenda for all meetings.}

\section{Team Communication Plan}

\wss{Issues on GitHub should be part of your communication plan.}

\section{Team Member Roles}

\wss{You should identify the types of roles you anticipate, like notetaker,
leader, meeting chair, reviewer.  Assigning specific people to those roles is
not necessary at this stage.  In a student team the role of the individuals will
likely change throughout the year.}

\section{Workflow Plan}

\begin{itemize}
	\item How will you be using git, including branches, pull request, etc.?
	\item How will you be managing issues, including template issues, issue
	classification, etc.?
  \item Use of CI/CD
\end{itemize}

\section{Project Decomposition and Scheduling}

\begin{itemize}
  \item How will you be using GitHub projects?
  \item Include a link to your GitHub project
\end{itemize}

\wss{How will the project be scheduled?  This is the big picture schedule, not
details. You will need to reproduce information that is in the course outline
for deadlines.}

\section{Proof of Concept Demonstration Plan}


As with most projects, there are certain risks that could affect how successful ours will be. Below are the main risks we have identified and plan to address:

\subsubsection*{Data Access \& API Limitations}\vspace{-0.5em}
One of the risks in our project is accessing the correct data through APIs. We may face challenges finding an API that provides the information we need in the way we require. In addition, the API we choose could be unreliable or return data in a format that is hard to process, such as poorly structured JSON or XML. There may also be restrictions, such as requiring special keys or authentication, and many APIs limit how many requests can be made within a given time. Any of these issues could make it difficult to collect enough usable data for our tool.
\subsubsection*{Integration Risk}\vspace{-0.5em}
Another risk is that even if each component works individually, combining them into one complete workflow could cause unexpected issues. For instance, data collected from the API might not store properly in the database, or the way the data is structured in the database may not connect smoothly with the visualization tool. These kinds of problems could lead to delays and require extra adjustments during development.
\subsubsection*{Expert Access \& Library Packages}\vspace{-0.5em}
Some packages may not be easy to access, as they could be private, outdated, or require a paid license that we do not have. Even if the package is available, the documentation may be very limited, which makes it harder to understand or use correctly. There is also the chance that these packages have complicated dependencies that are difficult to install or set up in the right environment. These challenges could slow down the evaluation process or limit what experts are able to review.

\subsubsection*{Data Management \& Database Integration}\vspace{-0.5em}
Databases are widely used, but there are still risks when applying them to our specific case. One challenge is designing a schema that fits the data we collect from APIs, since the structure may not align well with a relational database. There is also the possibility of data integrity issues, such as duplicate or missing values, which could lead to an unreliable dataset. In addition, if the amount of data grows larger than expected, the database may struggle with performance and make queries too slow for our application.

\subsubsection*{Visualization \& Graphing}\vspace{-0.5em}
Another risk is related to how we present the final results. If the data cannot be shown in a clear and effective way, the value of the tool is greatly reduced. The graphing library we choose might not fully support our data structure or may not integrate well with the rest of the system. There is also the chance that creating complex graphs with larger datasets could affect performance and make the tool slow to use. Finally, the library we select may not provide the types of charts or visualizations that best highlight the comparisons we need to show.

\subsubsection*{Lack of Prior Research}\vspace{-0.5em}
Another risk is the limited amount of prior research in this area. Without established methodologies to follow, we could end up spending extra time figuring out approaches from scratch. There is also the chance that we develop a tool or process that turns out to be very similar to something that already exists but is not well documented. This could reduce the originality of our work and make it harder to show the unique value of our project.
\subsubsection*{Subjectivity of Evaluation Criteria}\vspace{-0.5em}
Some evaluation criteria may rely too much on human judgment. Qualities such as usability, understandability, or maintainability are difficult to measure in a purely objective way, and different evaluators may score them differently. This could reduce the reproducibility and reliability of our results, especially if the tool is later applied in other contexts or by different teams.

\subsubsection*{Poor Data Quality}\vspace{-0.5em}
The data from public repositories may be incomplete, inconsistent, or outdated. This could make evaluation harder and lead to unreliable results, reducing the overall usefulness of our tool.

\vspace{1em} %
\noindent
To address these risks, our proof of concept will focus on showing that the main components of the system can work together in practice. For the risk of data access and API limitations, we will build a small script that connects to the intended API endpoint, displays the raw response, and then parses the data into a structured format such as a Python dictionary. The parsed data will then be inserted into a database and used to generate a simple visualization, confirming that the entire pipeline from retrieving external data to processing, storing, and displaying it functions as expected. To address the risk of library packages, we will test the installation of the suggested packages in our local environment to confirm they are accessible and usable. We will also review their documentation in advance, and if issues arise, discuss alternatives with our supervisor. For the risks around database and graphing, we will demonstrate that parsed data can be stored in a suitable schema, retrieved through simple queries, and displayed in a basic graph using our chosen visualization tool. To address the risk of limited prior research, we will review existing academic work, tools, and evaluation methods early in the project to ensure our approach is original and provides clear value, while confirming our direction with the supervisors. For the risk of subjectivity in evaluation criteria, we will design clear rubrics for factors such as usability or maintainability and test them by having multiple team members score the same library during the PoC. We will compare results for consistency, keep short notes to explain each score, and confirm our method with the supervisor before applying it more broadly. Finally, to address the risk of poor data quality, we will test the pipeline with a small sample of incomplete or inconsistent data to confirm that the system can detect missing or unreliable fields and either process or flag them instead of failing silently.
\section{Expected Technology}

\wss{What programming language or languages do you expect to use?  What external
libraries?  What frameworks?  What technologies.  Are there major components of
the implementation that you expect you will implement, despite the existence of
libraries that provide the required functionality.  For projects with machine
learning, will you use pre-trained models, or be training your own model?  }

\wss{The implementation decisions can, and likely will, change over the course
of the project.  The initial documentation should be written in an abstract way;
it should be agnostic of the implementation choices, unless the implementation
choices are project constraints.  However, recording our initial thoughts on
implementation helps understand the challenge level and feasibility of a
project.  It may also help with early identification of areas where project
members will need to augment their training.}

Topics to discuss include the following:

\begin{itemize}
\item Specific programming language
\item Specific libraries
\item Pre-trained models
\item Specific linter tool (if appropriate)
\item Specific unit testing framework
\item Investigation of code coverage measuring tools
\item Specific plans for Continuous Integration (CI), or an explanation that CI
  is not being done
\item Specific performance measuring tools (like Valgrind), if
  appropriate
\item Tools you will likely be using?
\end{itemize}

\wss{git, GitHub and GitHub projects should be part of your technology.}

\section{Coding Standard}

\wss{What coding standard will you adopt?}

\newpage{}

\section*{Appendix --- Reflection}

\wss{Not required for CAS 741}

The purpose of reflection questions is to give you a chance to assess your own
learning and that of your group as a whole, and to find ways to improve in the
future. Reflection is an important part of the learning process.  Reflection is
also an essential component of a successful software development process.  

Reflections are most interesting and useful when they're honest, even if the
stories they tell are imperfect. You will be marked based on your depth of
thought and analysis, and not based on the content of the reflections
themselves. Thus, for full marks we encourage you to answer openly and honestly
and to avoid simply writing ``what you think the evaluator wants to hear.''

Please answer the following questions.  Some questions can be answered on the
team level, but where appropriate, each team member should write their own
response:


\begin{enumerate}
    \item Why is it important to create a development plan prior to starting the
    project?
    \item In your opinion, what are the advantages and disadvantages of using
    CI/CD?
    \item What disagreements did your group have in this deliverable, if any,
    and how did you resolve them?
\end{enumerate}

\newpage{}

\section*{Appendix --- Team Charter}

\wss{borrows from
\href{https://engineering.up.edu/industry_partnerships/files/team-charter.pdf}
{University of Portland Team Charter}}

\subsection*{External Goals}

\wss{What are your team's external goals for this project? These are not the
goals related to the functionality or quality fo the project.  These are the
goals on what the team wishes to achieve with the project.  Potential goals are
to win a prize at the Capstone EXPO, or to have something to talk about in
interviews, or to get an A+, etc.}

\subsection*{Attendance}

\subsubsection*{Expectations}

\wss{What are your team's expectations regarding meeting attendance (being on
time, leaving early, missing meetings, etc.)?}

\subsubsection*{Acceptable Excuse}

\wss{What constitutes an acceptable excuse for missing a meeting or a deadline?
What types of excuses will not be considered acceptable?}

\subsubsection*{In Case of Emergency}

\wss{What process will team members follow if they have an emergency and cannot
attend a team meeting or complete their individual work promised for a team
deliverable?}

\subsection*{Accountability and Teamwork}

\subsubsection*{Quality} 

\wss{What are your team's expectations regarding the quality
of team members' preparation for team meetings and the quality of the
deliverables that members bring to the team?}

\subsubsection*{Attitude}

\wss{What are your team's expectations regarding team members' ideas,
interactions with the team, cooperation, attitudes, and anything else regarding
team member contributions?  Do you want to introduce a code of conduct?  Do you
want a conflict resolution plan?  Can adopt existing codes of conduct.}

\subsubsection*{Stay on Track}

\wss{What methods will be used to keep the team on track? How will your team
ensure that members contribute as expected to the team and that the team
performs as expected? How will your team reward members who do well and manage
members whose performance is below expectations?  What are the consequences for
someone not contributing their fair share?}

\wss{You may wish to use the project management metrics collected for the TA and
instructor for this.}

\wss{You can set target metrics for attendance, commits, etc.  What are the
consequences if someone doesn't hit their targets?  Do they need to bring the
coffee to the next team meeting?  Does the team need to make an appointment with
their TA, or the instructor?  Are there incentives for reaching targets early?}

\subsubsection*{Team Building}

\wss{How will you build team cohesion (fun time, group rituals, etc.)? }

\subsubsection*{Decision Making} 

\wss{How will you make decisions in your group? Consensus?  Vote? How will you
handle disagreements? }

\end{document}