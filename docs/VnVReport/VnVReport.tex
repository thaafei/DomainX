\documentclass[12pt, titlepage]{article}

\usepackage{booktabs}
\usepackage{tabularx}
\usepackage{hyperref}
\hypersetup{
    colorlinks,
    citecolor=black,
    filecolor=black,
    linkcolor=red,
    urlcolor=blue
}
\usepackage[round]{natbib}

\input{../Comments}
\input{../Common}

\begin{document}

\title{Verification and Validation Report: \progname} 
\author{\authname}
\date{\today}
	
\maketitle

\pagenumbering{roman}

\section{Revision History}

\begin{tabularx}{\textwidth}{p{3cm}p{2cm}X}
\toprule {\bf Date} & {\bf Version} & {\bf Notes}\\
\midrule
Date 1 & 1.0 & Notes\\
Date 2 & 1.1 & Notes\\
\bottomrule
\end{tabularx}

~\newpage

\section{Symbols, Abbreviations and Acronyms}

\renewcommand{\arraystretch}{1.2}
\begin{tabular}{l l} 
  \toprule		
  \textbf{symbol} & \textbf{description}\\
  \midrule 
  T & Test\\
  \bottomrule
\end{tabular}\\

\wss{symbols, abbreviations or acronyms -- you can reference the SRS tables if needed}

\newpage

\tableofcontents

\listoftables %if appropriate

\listoffigures %if appropriate

\newpage

\pagenumbering{arabic}

This document ...

\section{Functional Requirements Evaluation}

\section{Nonfunctional Requirements Evaluation}

\subsection{Usability}
		
\subsection{Performance}

\subsection{etc.}

\section{Unit Testing}
The following unit tests concern behaviour-hiding modules that are implemented by the tool. Modules implemented by external libraries are assumed to be tested themselves and therefore don't need to be unit tested. 
\subsection{Data Edit Module}
\subsection{User Authentication Module}
\subsection{User Role Access Module}
\subsection{User Page Module}
\subsection{Automated Metrics Module}
\subsection{Domains Page Module}
  \begin{enumerate}
    \item \texttt{test\_create\_domain\_success} \label{test-create-domain-success}
      \\Initial State: Create user and fixture of API client
      \\Input: Create payload containing domain name, description, and \texttt{creator\_ids} with the creator and call POST
      \\Expected Output: Response returns status 201 for created, and querying for the domain object contains the same as in the payload
      \\Actual Output: Matches expected
      \\Result: Pass
    \item \texttt{test\_create\_domain\_missing\_domain\_name} \label{test-create-domain-missing-domain-name}
      \\Initial State: Create fixture of API client
      \\Input: Create payload with only description and call POST
      \\Expected Output: Response returns 400 bad request and informs of missing field
      \\Actual Output: Matches expected, data returns dictionary with \texttt{domain\_name} and value of "This field is required"
      \\Result: Pass
    \item \texttt{test\_create\_domain\_missing\_description} \label{test-create-domain-missing-description}
      \\Initial State: Create fixture of API client
      \\Input: Create payload with only domain name and call POST
      \\Response returns 400 bad request and informs of missing field
      \\Actual Output: Matches expected, data returns dictionary with "description" and value of "Description field is required"
      \\Result: Pass
    \item \texttt{test\_list\_domains} \label{test-list-domains}
      \\Initial State: Create fixture of API client and create two domain objects
      \\Input: Calling GET for domain
      \\Expected Output: Response 200 OK with list of domains matching the created domains
      \\Actual Output: Matches expected
      \\Result: Pass
    \item \texttt{test\_update\_domain\_sets\_creators} \label{test-update-domain-sets-creators}
      \\Initial State: Create domain object with only \texttt{domain\_name} and description and fixture of API client
      \\Input: Create two users and add them to a payload with updated domain name and description. Then call POST with the domain ID
      \\Expected Output: Response 200, domain should contain the new information and be linked to the two users
      \\Actual Output: Matches expected
      \\Result: Pass
    \item \texttt{test\_delete\_domain} \label{test-delete-domain}
      \\Initial State: Create fixture of API client and create a domain object
      \\Input: Call DELETE with the domain ID
      \\Expected Output: Querying for the domain ID returns None
      \\Actual Output: Matches Expected
      \\Result: Pass
  \end{enumerate}
\subsection{Comparison Module}
\subsection{Configuration Module}
\subsection{Ranking Algorithm Module}
\subsection{Database Persistence Module}
  \begin{enumerate}
    \item \texttt{test\_domain\_str\_returns\_name} \label{test-domain-str-returns-name}
      \\Initial State: N/A
      \\Input: Directly creating a new domain object with \texttt{domain\_name} "Alpha" and description "desc"
      \\Expected Output: Calling newly created object returns the domain name "Alpha"
      \\Actual Output: Returned "Alpha"
      \\Result: Pass
    \item \texttt{test\_get\_domain\_ID\_returns\_string\_uuid} \label{test-get-domain-ID-returns-string-uuid}
      \\Initial State: N/A
      \\Input: Directly creating a new domain object with \texttt{domain\_name} "Beta" and description "desc"
      \\Expected Output: Calling \texttt{get\_domain\_ID()} on new domain returns the UUID
      \\Actual Output: Method matches \texttt{domain\_id} parameter and is instance of string
      \\Result: Pass
    \item \texttt{test\_defaults\_and\_blank\_fields}\label{test-defaults-and-blank-fields}
      \\Initial State: N/A
      \\Input: Directly creating a new domain object with \texttt{domain\_name} "Gamma"
      \\Expected Output: Initializes domain object with default fields \texttt{{published: False, paper\_name: None, paper\_url: "", category\_weights: {}}}
      \\Actual Output: Output matches expected
      \\Result: Pass
    \item \texttt{test\_unique\_domain\_name\_constraint} \label{test-unique-domain-name-constraint}
      \\Initial State: Domain object with \texttt{domain\_name} "Delta" and description "one"
      \\Input: Directly creating a new domain object with \texttt{domain\_name} "Delta" and description "two"
      \\Expected Output: Throws integrity error due to duplicate domain names
      \\Actual Output: Matches expected
      \\Result: Pass
    \item \texttt{test\_creators\_many\_to\_many\_assignment} \label{test-creators-many-to-many-assignment}
      \\Initial State: Create two users with admin role
      \\Input: Create two domain objects with separate names, add both creators to each domain.
      \\Expected Output: Filtering in each domain's creators by the id of the creator, both creators should exist
      \\Actual Output: Matches expected
      \\Result: Pass
  \end{enumerate}
\section{Changes Due to Testing}

\wss{This section should highlight how feedback from the users and from 
the supervisor (when one exists) shaped the final product.  In particular 
the feedback from the Rev 0 demo to the supervisor (or to potential users) 
should be highlighted.}

\section{Automated Testing}
		
\section{Trace to Requirements}
		
\section{Trace to Modules}		

\section{Code Coverage Metrics}

\bibliographystyle{plainnat}
\bibliography{../../refs/References}

\newpage{}
\section*{Appendix --- Reflection}

The information in this section will be used to evaluate the team members on the
graduate attribute of Reflection.

\input{../Reflection.tex}

\begin{enumerate}
  \item What went well while writing this deliverable? 
  \item What pain points did you experience during this deliverable, and how
    did you resolve them?
  \item Which parts of this document stemmed from speaking to your client(s) or
  a proxy (e.g. your peers)? Which ones were not, and why?
  \item In what ways was the Verification and Validation (VnV) Plan different
  from the activities that were actually conducted for VnV?  If there were
  differences, what changes required the modification in the plan?  Why did
  these changes occur?  Would you be able to anticipate these changes in future
  projects?  If there weren't any differences, how was your team able to clearly
  predict a feasible amount of effort and the right tasks needed to build the
  evidence that demonstrates the required quality?  (It is expected that most
  teams will have had to deviate from their original VnV Plan.)
\end{enumerate}

\end{document}