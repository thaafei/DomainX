\subsection*{Fei}
\begin{enumerate}
  \item \textbf{What went well while writing this deliverable?} \\
  Each team member knew what requirements were, and would work on the sections we decided during the team meeting.
  \item \textbf{What pain points did you experience during this deliverable, and how did
  you resolve them?} \\
  Issues getting the rest of the team to review my work and getting their work up for review. Leaving it hard for me to review their work if it's uploaded so late. This makes it hard to structure PRs, since chances are no one is reviewing my PR even with the team assigned as reviewers, and I can't tell if they seen the comments I left on their PR either. Still wasn't able to completely resolve the issues of reviewing, hopefully a team meeting specifically on the expectations of reviews will clear this up.
  \item \textbf{How many of your requirements were inspired by speaking to your client(s) or their proxies (e.g. your peers, stakeholders, potential users)?} \\
  Due to the unique nature of our project, almost all the requirements were inspired by our client, since we are the clients. After reflecting on what we wanted to see when we use the tool in writing the paper, we were able to easily think of requirements.
  \item \textbf{Which of the courses you have taken, or are currently taking, will help your team to be successful with your capstone project.} \\
  3RA3 has definitely helped, as well as 4HC3, in terms of understanding what a good layout would be (Users are drawn to colours in the middle of their vision, but more sensitive to motion on the peripherals)
  \item \textbf{What knowledge and skills will the team collectively need to acquire to 
  successfully complete this capstone project?  Examples of possible knowledge
  to acquire include domain specific knowledge from the domain of your
  application, or software engineering knowledge, mechatronics knowledge or
  computer science knowledge.  Skills may be related to technology, or writing,
  or presentation, or team management, etc.  You should look to identify at
  least one item for each team member.} \\
  I think the team needs to acquire system design knowledge. Although we've all had experience working adding/updating features of pre-existing projects, starting a completely new one and having to think of all the design requirements will be hard. Additionally, due to the research section, we'll also need to get research, and by expansion, writing knowledge.
  \item \textbf{For each of the knowledge areas and skills identified in the previous
  question, what are at least two approaches to acquiring the knowledge or
  mastering the skill?  Of the identified approaches, which will each team
  member pursue, and why did they make this choice?} \\
  For my, to pursue the system design knowledge, I believe the best way is to just do it. Early consideration of the design of the project, as well as determining the systems that'll need integrating is crucial. Luckily, we've already started doing that so the rest will just be trial and error. Also reviewing 3RA3 content when needed.
  For the research/writing knowledge, I'll be reviewing the methodology research paper as well as the existing papers on other domains. As well as consulting with our supervisor during the writing process.
\end{enumerate}
\subsection*{Haniye}
\begin{enumerate}
  \item What went well while writing this deliverable? 
  \item What pain points did you experience during this deliverable, and how did
  you resolve them?
  \item How many of your requirements were inspired by speaking to your
  client(s) or their proxies (e.g. your peers, stakeholders, potential users)?
  \item Which of the courses you have taken, or are currently taking, will help
  your team to be successful with your capstone project.
  \item What knowledge and skills will the team collectively need to acquire to
  successfully complete this capstone project?  Examples of possible knowledge
  to acquire include domain specific knowledge from the domain of your
  application, or software engineering knowledge, mechatronics knowledge or
  computer science knowledge.  Skills may be related to technology, or writing,
  or presentation, or team management, etc.  You should look to identify at
  least one item for each team member.
  \item For each of the knowledge areas and skills identified in the previous
  question, what are at least two approaches to acquiring the knowledge or
  mastering the skill?  Of the identified approaches, which will each team
  member pursue, and why did they make this choice?
\end{enumerate}
\subsection*{Ghena}

\begin{enumerate}
  \item \textbf{What went well while writing this deliverable?} 
  Before meeting the stakeholder, I read over what I sections covered and wrote what my assumptions were for the requirements. During the meeting, I showed the professor the use case diagram and some of the requirements based on my assumption of how things would work. However, after talking to him there was a few changes that needed to be made, such as it not having to be too secure, allowing all researchers to change any domain, just ensuring that only researchers can modify the data. Also ensured that the stakeholder had an agenda with the questions that we wanted to ask the day before the meeting. 
  \item \textbf{What pain points did you experience during this deliverable, and how did
  you resolve them?}
  During our team meeting, we split up the tasks, however we did not realize that we missed some sections until we combined our work. We also didn't review the whole document until the day of summation. My computer wasn't turning on, making it harder to participate in the conversations happening on discord and running make on campus laptops.
  \item \textbf{How many of your requirements were inspired by speaking to your
  client(s) or their proxies (e.g. your peers, stakeholders, potential users)?}
  Most of the requirements were inspired from our stakeholder, especially the functional requirements. The security requirements were obtained by the team members, since the stakeholder wasn't thinking too heavily on that section. 
  \item \textbf{Which of the courses you have taken, or are currently taking, will help
  your team to be successful with your capstone project.}
  4HC3 - Human Computer Interfaces, survey stakeholder for to understand how they would like the website to look and how they will be interacting with it, as well as analyze our designs and find their strengths and weaknesses are, 3DB3 - Databases, design a relational database schema based on requirements, 3A04 - Software Design III - Large System Design, analyze what type of design infostructure needed for this project, 2AA4 - Software Design I - Introduction to Software Development, building requirement doc. 
  \item \textbf{What knowledge and skills will the team collectively need to acquire to
  successfully complete this capstone project?  Examples of possible knowledge
  to acquire include domain specific knowledge from the domain of your
  application, or software engineering knowledge, mechatronics knowledge or
  computer science knowledge.  Skills may be related to technology, or writing,
  or presentation, or team management, etc.  You should look to identify at
  least one item for each team member.}
  The team will need to look into the created research paper as well as the steps taken in this area, so reading the Methodology for Assessing the State of the Practice for Domain X paper. We also need to brush-up on design architectures for software systems, to ensure it follows the SOLID principle. 
  \item \textbf{For each of the knowledge areas and skills identified in the previous
  question, what are at least two approaches to acquiring the knowledge or
  mastering the skill?  Of the identified approaches, which will each team
  member pursue, and why did they make this choice?}
  Read over the Methodology for Assessing the State of the Practice for Domain X paper and start with creating a list of libraries to analyze. Read over previous course notes.
\end{enumerate}
\subsection*{Awurama}

\begin{enumerate}
  \item What went well while writing this deliverable? 
  \item What pain points did you experience during this deliverable, and how did
  you resolve them?
  \item How many of your requirements were inspired by speaking to your
  client(s) or their proxies (e.g. your peers, stakeholders, potential users)?
  \item Which of the courses you have taken, or are currently taking, will help
  your team to be successful with your capstone project.
  \item What knowledge and skills will the team collectively need to acquire to
  successfully complete this capstone project?  Examples of possible knowledge
  to acquire include domain specific knowledge from the domain of your
  application, or software engineering knowledge, mechatronics knowledge or
  computer science knowledge.  Skills may be related to technology, or writing,
  or presentation, or team management, etc.  You should look to identify at
  least one item for each team member.
  \item For each of the knowledge areas and skills identified in the previous
  question, what are at least two approaches to acquiring the knowledge or
  mastering the skill?  Of the identified approaches, which will each team
  member pursue, and why did they make this choice?
\end{enumerate}
